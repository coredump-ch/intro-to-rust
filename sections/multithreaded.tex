\section{Multithreaded Programming}

\begin{frame}{We'll make this short}
\begin{itemize}
  \item The Rust compiler does not know about concurrency
  \item Everything works based on the three rules\footnote{Slide~\ref{threerules}}
  \item We'll step through an example
\end{itemize}
\end{frame}

\begin{frame}[fragile]{Threads}
\begin{minted}{rust}
let t1 = std::thread::spawn(|| { return 23; });
let t2 = std::thread::spawn(|| { return 19; });

let v1 = t1.join().unwrap();
let v2 = t2.join().unwrap();

println!("{} + {} = {}", v1, v2, v1 + v2);
\end{minted}
\end{frame}

\begin{frame}[fragile]{Shared Data}
\begin{minted}{rust}
let mut data = vec![0];
let t1 = thread::spawn(|| { data.push(19); });
\end{minted}
\sep
\begin{minted}[fontsize=\footnotesize]{text}
error[E0373]: closure may outlive the current function, but it borrows `data`,
              which is owned by the current function
 --> src/main.rs:5:28
  |
5 |     let t1 = thread::spawn(|| { data.push(19); });
  |                            ^^   ---- `data` is borrowed here
  |                            |
  |                            may outlive borrowed value `data`
  |
note: function requires argument type to outlive `'static`
 --> src/main.rs:5:14
  |
5 |     let t1 = thread::spawn(|| { data.push(19); });
  |              ^^^^^^^^^^^^^^^^^^^^^^^^^^^^^^^^^^^^
\end{minted}
\end{frame}

\begin{frame}[fragile]{Shared Data (2) -- Move data}
The compiler also gives us a helpful hint:
\begin{minted}[fontsize=\footnotesize]{text}
help: to force the closure to take ownership of `data` (and any
      other referenced variables), use the `move` keyword
  |
5 |     let t1 = thread::spawn(move || { data.push(19); });
  |                            ^^^^^^^
\end{minted}
Let's move the \texttt{data} into the Thread.
\begin{minted}{rust}
let mut data = vec![0];
let t1 = thread::spawn(move || { data.push(19); });
\end{minted}
\end{frame}

\begin{frame}[fragile]{Shared Data (3) -- Outside Access}
But now we can't access it anymore..
\begin{minted}{rust}
let mut data = vec![0];
let t1 = thread::spawn(move || { data.push(19); });
t1.join().unwrap();
println!("Data: {:?}", data);
\end{minted}
\sep
\begin{minted}[fontsize=\footnotesize]{text}
error[E0382]: borrow of moved value: `data`
4 |     let mut data = vec![0];
  |         -------- move occurs because `data` has type `std::vec::Vec<i32>`,
  |                  which does not implement the `Copy` trait
5 |     let t1 = thread::spawn(move || { data.push(19); });
  |                            -------   ---- variable moved due to use
  |                            |              in closure
  |                            value moved into closure here
7 |     println!("Data: {:?}", data);
  |                            ^^^^ value borrowed here after move
\end{minted}
\end{frame}

\begin{frame}[fragile]{Shared Data (4) -- Arcs}
Atomic reference counting to the rescue!
\begin{minted}{rust}
let data = Arc::new(vec![0]);

let data2 = data.clone();
let t1 = thread::spawn(move || {
    println!("Data2: {:?}", data2);
});

t1.join().unwrap();
println!("Data: {:?}", data);
\end{minted}
\sep
\begin{minted}[fontsize=\footnotesize]{text}
Data2: [0]
Data: [0]
\end{minted}
\end{frame}

\begin{frame}[fragile]{Shared Data (5) -- Mutate?}
\begin{minted}{rust}
let data = Arc::new(vec![0]);

let data2 = data.clone();
let t1 = thread::spawn(move || { data2.push(1); });

t1.join().unwrap();
println!("Data: {:?}", data);
\end{minted}
\sep
\begin{minted}[fontsize=\footnotesize]{text}
error[E0596]: cannot borrow data in a `&` reference as mutable
 --> src/main.rs:8:38
  |
8 |     let t1 = thread::spawn(move || { data2.push(1); });
  |                                      ^^^^^ cannot borrow as mutable
\end{minted}
\end{frame}

\begin{frame}[fragile]{Shared Data (6) -- Arc + Mutex}
\begin{minted}{rust}
let data = Arc::new(Mutex::new(vec![0]));

let data2 = data.clone();
let t1 = thread::spawn(move || {
    let mut guard = data2.lock().unwrap();
    guard.push(1);
});

t1.join().unwrap();
println!("Data: {:?}", data.lock().unwrap());
\end{minted}
\sep
\begin{minted}[fontsize=\footnotesize]{text}
Data: [0, 1]
\end{minted}
\end{frame}

\begin{frame}[fragile]{Shared Data (7) -- Multiple Threads}
Now we can also create multiple threads.
\begin{minted}[fontsize=\footnotesize]{rust}
...
let data2 = data.clone();
let t1 = thread::spawn(move || {
    let mut guard = data2.lock().unwrap();
    guard.push(1);
});

let data3 = data.clone();
let t2 = thread::spawn(move || {
    let mut guard = data3.lock().unwrap();
    guard.push(2);
});
...
\end{minted}
\sep
\begin{minted}[fontsize=\footnotesize]{text}
Data: [0, 1, 2]
\end{minted}
\end{frame}

\begin{frame}[fragile]{Channels}
To simplify communication with Threads, you can also use channels:
\begin{minted}{rust}
use std::sync::mpsc::channel;
\end{minted}
Signature:
\begin{minted}{rust}
fn channel<T>() -> (Sender<T>, Receiver<T>)
\end{minted}
\end{frame}

\begin{frame}[fragile]{Async / Await}
Brand new: Startin with 1.39, Rust has built-in async / await syntax!
\begin{minted}{rust}
async fn first_function() -> u32 { .. }
\end{minted}

This function implicitly returns a \texttt{Future<u32>}. To wait
for the result in another async function:

\begin{minted}{rust}
async fn another_function() {
    let result: u32 = first_function().await;
}
\end{minted}

...but that's out of scope for this talk.
\end{frame}
