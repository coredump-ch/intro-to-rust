\subsection{Promise 1: No null pointer dereferences}

%%%

\begin{frame}{Promise 1: No null pointer dereferences}

\begin{block}{Null pointers are useful.}
They can indicate the absence of optional information.\\
They can indicate failures.\\
But they can introduce severe bugs.
\end{block}
\vspace{1em}
\pause
\begin{block}{Rust separates the concept of a pointer from the concept of an\\
		optional or error value.}
	Optional values are handled by \texttt{Option<T>}.\\
  Error values are handled by \texttt{Result<T, E>}.\\
	Many helpful tools to do error handling.
\end{block}

\end{frame}

%%%

\begin{frame}[fragile]{You already saw \texttt{Option<T>}}
\begin{minted}{rust}
fn safe_div(n: i32, d: i32) -> Option<i32> {
    if d == 0 {
        return None;
    }
    Some(n / d)
}
\end{minted}
\pause
But what if you want to return an error, not just \texttt{None}?
\end{frame}

%%%

\begin{frame}[fragile]{There's also \texttt{Result<T, E>}}
\begin{minted}{rust}
enum Result<T, E> {
    Ok(T),
    Err(E)
}
\end{minted}
\end{frame}

%%%

\begin{frame}[fragile]{How to use \texttt{Result}s:}
\begin{minted}{rust}
enum Error {
    DivisionByZero,
}

fn safe_div(n: i32, d: i32) -> Result<i32, Error> {
    if d == 0 {
        return Err(Error::DivisionByZero);
    }
    Ok(n / d)
}
\end{minted}
\pause
It's good practice to define your own error types instead of using strings.
\end{frame}

%%%

\begin{frame}[fragile]{But \texttt{Result} can get tedious...}
\begin{minted}{rust}
fn do_calc(a: i32, b: i32, c: i32) -> Result<i32, Error> {
    let tmp = match safe_div(a, b) {
        Ok(val) => val,
        Err(e) => return Err(e),
    };
    let res = match save_div(tmp, c) {
        Ok(val) => val,
        Err(e) => return Err(e),
    };
    Ok(res)
}
\end{minted}
\end{frame}

%%%

\begin{frame}[fragile]{Ergonomic error handling with the \texttt{?} operator}
\begin{minted}{rust}
fn do_calc(a: i32, b: i32, c: i32) -> Result<i32, Error> {
    let tmp = safe_div(a, b)?;
    let res = safe_div(tmp, c)?;
    Ok(res)
}

// ...or chained

fn do_calc(a: i32, b: i32, c: i32) -> Result<i32, Error> {
    Ok(safe_div(safe_div(a, b)?, c)?)
}
\end{minted}
\pause
Note: Error signature must match!
\end{frame}

%%%

\begin{frame}[fragile]{Mapping Errors}
What if the signature does not match?
\pause
Then we can use map\_err():
\begin{minted}{rust}
fn do_subcalc() -> Result<i32, String> { ... }

fn do_calc() -> Result<i32, Error> {
    let val = do_subcalc()
        .map_err(|msg| {
            println!("Error: {}", msg);
            Error::CalcFailed
        })?;
    Ok(val + 1)
}
\end{minted}
\end{frame}

%%%

\begin{frame}[fragile]{Mapping Errors: A closer look}
\begin{minted}{rust}
let mapped = do_subcalc().map_err(|msg| Error::CalcFailed);
\end{minted}
...is the same as...
\begin{minted}{rust}
let mapped = match do_subcalc() {
    Ok(val) => Ok(val),
    Err(msg) => Err(Error::CalcFailed),
}
\end{minted}
\end{frame}

%%%

\begin{frame}[fragile]{Mapping Errors: Implicit conversion}
The \texttt{?} operator even allows for implicit conversion if the
\texttt{From<OtherError>} trait is implemented for your error type, or if the
other error type implements \texttt{Into<YourError>}.

\begin{minted}{rust}
enum MyError {
    Io(io::Error),
    GenericMessage(String),
}

impl std::convert::From<io::Error> for MyError {
    fn from(error: io::Error) -> Self {
        MyError::Io(error)
    }
}
\end{minted}
\end{frame}

%%%

\begin{frame}[fragile]{Error handling for the lazy}
If an error is impossible to occur, or if you don't want to handle the error at
all, you can \texttt{unwrap()} or \texttt{expect(msg)} the result.

\begin{minted}{rust}
let val: u32 = "42".parse().unwrap();
let val: u32 = "42".parse().expect("Failed to parse integer");
\end{minted}

If the result is successful, it will return the contents. Otherwise,
the program will panic.

The \texttt{unwrap} and \texttt{expect} methods works for \texttt{Option}s too!
\end{frame}
