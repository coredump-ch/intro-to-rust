\section{Memory Safety in Rust}

%%%

\begin{frame}{Three Key Promises}

To guarantee memory safety, Rust gives us three key promises:

\begin{itemize}
	\item No null pointer dereferences
		\pause
		\begin{itemize}
			\item There are no null pointers in safe Rust
			\item For error handling and control flow, \texttt{Option} and
				\texttt{Result} types are used.
		\end{itemize}
	\pause
	\item No dangling pointers
		\pause
		\begin{itemize}
			\item The concepts of "ownership", "borrowing" and "lifetimes" prevent the
				use of uninitialized or freed pointers
		\end{itemize}
	\pause
	\item No buffer overruns
		\pause
		\begin{itemize}
			\item There's no pointer arithmetic in safe Rust
			\item Arrays in Rust are not just pointers
			\item There are runtime bounds checks for indexing
			\item But most stdlib functions use iterators, which are checked at
				compile time
		\end{itemize}
\end{itemize}

\end{frame}

%%%

\subsection{Promise 1: No null pointer dereferences}

%%%

\begin{frame}{Promise 1: No null pointer dereferences}

\begin{block}{Null pointers are useful.}
They can indicate the absence of optional information.\\
They can indicate failures.\\
\pause
But they can introduce severe bugs.
\end{block}
\vspace{1em}
\pause
\begin{block}{Rust separates the concept of a pointer from the concept of an\\
		optional or error value.}
	Optional values are handled by \texttt{Option<T>}.\\
  Error values are handled by \texttt{Result<T, E>}.\\
	Many helpful tools to do error handling.
\end{block}

\end{frame}

%%%

\begin{frame}[fragile]{You already saw \texttt{Option<T>}}
\begin{minted}{rust}
fn safe_div(n: i32, d: i32) -> Option<i32> {
    if d == 0 {
        return None;
    }
    Some(n / d)
}
\end{minted}
\end{frame}

%%%

\begin{frame}[fragile]{There's also \texttt{Result<T, E>}}
\begin{minted}{rust}
enum Result<T, E> {
    Ok(T),
    Err(E)
}
\end{minted}
\end{frame}

%%%

\begin{frame}[fragile]{How to use \texttt{Result}s:}
\begin{minted}{rust}
enum Error {
    DivisionByZero,
}

fn safe_div(n: i32, d: i32) -> Result<i32, Error> {
    if d == 0 {
        return Err(Error::DivisionByZero);
    }
    Ok(n / d)
}
\end{minted}
\end{frame}

%%%

\begin{frame}[fragile]{But \texttt{Result} can get tedious...}
\begin{minted}{rust}
fn do_calc() -> Result<i32, String> {
    let a = match do_subcalc1() {
        Ok(val) => val,
        Err(msg) => return Err(msg),
    }
    let b = match do_subcalc2() {
        Ok(val) => val,
        Err(msg) => return Err(msg),
    }
    Ok(a + b)
}
\end{minted}
\end{frame}

%%%

\begin{frame}[fragile]{Ergonomic error handling with the \texttt{try!} macro}
\begin{minted}{rust}
fn do_calc() -> Result<i32, String> {
    let a = try!(do_subcalc1());
    let b = try!(do_subcalc2());
    Ok(a + b)
}
\end{minted}
\end{frame}

%%%

\begin{frame}[fragile]{Mapping Errors}
\begin{minted}{rust}
fn do_subcalc() -> Result<i32, String> { ... }
fn do_calc() -> Result<i32, Error> {
    let res = do_subcalc();
    let mapped = res.map_err(|msg| {
        println!("Error: {}", msg);
        Error::CalcFailed
    });
    let val = try!(mapped);
    Ok(val + 1)
}
\end{minted}
\end{frame}

%%%

\begin{frame}[fragile]{Mapping Errors: A closer look}
\begin{minted}{rust}
let mapped = res.map_err(|msg| Error::CalcFailed);
\end{minted}
...is the same as...
\begin{minted}{rust}
let mapped = match res {
    Ok(val) => Ok(val),
    Err(msg) => Err(Error::CalcFailed),
}
\end{minted}
\end{frame}

%%%

\begin{frame}[fragile]{Other Combinator Methods}
Get the value from an \texttt{Option}.
\sep
\begin{minted}{rust}
Option.unwrap(self) -> T
Option.unwrap_or(self, def: T) -> T
Option.unwrap_or_else<F>(self, f: F) -> T
    where F: FnOnce() -> T
\end{minted}
\end{frame}

%%%

\begin{frame}[fragile]{Other Combinator Methods}
Map an \texttt{Option<T>} to \texttt{Option<U>} or \texttt{U}.
\sep
\begin{minted}{rust}
Option.map<U, F>(self, f: F) -> Option<U>
    where F: FnOnce(T) -> U
Option.map_or<U, F>(self, default: U, f: F) -> U
    where F: FnOnce(T) -> U
Option.map_or_else<U, D, F>(self, default: D, f: F) -> U
    where F: FnOnce(T) -> U, D: FnOnce() -> U
\end{minted}
\end{frame}

%%%

\begin{frame}[fragile]{Other Combinator Methods}

Convert an \texttt{Option} to a \texttt{Result}, mapping \texttt{Some(v)} to
\texttt{Ok(v)}\\and \texttt{None} to \texttt{Err(err)}.
\sep
\begin{minted}{rust}
Option.ok_or<E>(self, err: E)
    -> Result<T, E>
Option.ok_or_else<E, F>(self, err: F)
    -> Result<T, E>
    where F: FnOnce() -> E
\end{minted}
\end{frame}
